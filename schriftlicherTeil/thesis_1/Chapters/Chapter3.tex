% Chapter 3

\chapter{Implementation} % Main chapter title

\label{Chapter3}

We will now look upon the implementation of the aforementioned changes. Unlike to the original
implementation this is not a C library, but a pure Haskell library. This brings some advantages 
and one disadvantages. The disadvantage is the performance as discussed in \textbf{REF TO EVALUATION}.
For the costs of performance we gain a library that is easy to understand and extend. The original 
implementation is interwoven with the GHC runtime environment. Some STM functions are evoked by 
the scheduler to ensure the consistency. This makes the library sensitive to changes. 
To ensure the correctness of such a library is significantly harder than with a pure library, since
the compiler does not aid this porcess. In other words, the devlopment of a pure library is safer
and faster. Nevertheless, since the aim of this thesis is to optimize the current implementation,
it would be nice to have a C library of this implementation for realistic comparison, because 
the alternative implementation is not yet faster than the original one which we discuss is 
section \textbf{REF TO EVALUATION}. 