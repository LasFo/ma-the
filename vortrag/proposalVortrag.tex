\documentclass{beamer}
%\documentclass[trans]{beamer}

\usetheme[%pageofpages=/,% String used between the current page and the total page count.
%bullet=circle,% Use circles instead of squares for bullets.
titleline=true,% Show a line below the frame title.
alternativetitlepage=true,% Use the fancy title page.
%titlepagelogo=logo-polito,% Logo for the first page.
]{Torino} 

\usepackage{parcolumns}
\usepackage{listings} 
\usepackage{subfig}
\author{Lasse Folger}
\title{\huge Software Transaction Roll Back Avoidance}
\subtitle{Master Proposal Talk}
\date{\today}

\lstset{escapeinside=!!}
\begin{document}
  \begin{frame}[t,plain]
    \titlepage
  \end{frame}
  
  
  \begin{frame}
    \frametitle{Motivation}
    \lstinputlisting{ressources/accountMVar.hs}
  \end{frame}
  
  \begin{frame}[fragile]
    \frametitle{MVar}
    \fboxsep=0pt
    \noindent
    \begin{minipage}[t]{0.48\linewidth}
      Thread 1:
            \begin{figure}
       \begin{lstlisting}[frame=single]
transfer acc1 acc2 50
       \end{lstlisting}
      \end{figure}
\end{minipage}%
    \hfill%
    \begin{minipage}[t]{0.48\linewidth}
      Thread 2:      
      \begin{figure}
       \begin{lstlisting}[frame=single]
transfer acc2 acc1 50
       \end{lstlisting}
      \end{figure}
    \end{minipage}
\end{frame}

  \begin{frame}[fragile]
    \frametitle{MVar}
    \fboxsep=0pt
    \noindent
    \begin{minipage}[t]{0.48\linewidth}
      Thread 1:
            \begin{figure}
       \begin{lstlisting}[frame=single]
a1 <- takeMVar acc1 
a2 <- takeMVar acc2 
writeMVar acc1 (a1 - 50)
writeMVar acc2 (a2 + 50)
       \end{lstlisting}
      \end{figure}
\end{minipage}%
    \hfill%
    \begin{minipage}[t]{0.48\linewidth}
      Thread 2:      
      \begin{figure}
       \begin{lstlisting}[frame=single]
a1 <- takeMVar acc2 
a2 <- takeMVar acc1 
writeMVar acc2 (a1 - 50)
writeMVar acc1 (a2 + 50)
       \end{lstlisting}
      \end{figure}
    \end{minipage}
    \vfill
    \pause
    $\Rightarrow$ Deadlock
\end{frame}
  
  \begin{frame}
    \frametitle{Use Transactions}
    \lstinputlisting{ressources/accountTVar.hs}   
  \end{frame}
  
  \begin{frame}[fragile]
    \frametitle{TVar}
    \fboxsep=0pt
    \noindent
    \begin{minipage}[t]{0.48\linewidth}
      Thread 1:
      \begin{figure}
       \begin{lstlisting}[frame=single]
atomically $ 
  transfer acc1 acc2 50
       \end{lstlisting}
      \end{figure}
    \end{minipage}%
    \hfill%
    \begin{minipage}[t]{0.48\linewidth}
      Thread 2:
      \begin{figure}
       \begin{lstlisting}[frame=single]
atomically $ 
  transfer acc2 acc1 50
       \end{lstlisting}
      \end{figure}
    \end{minipage}
    \vfill
    \pause
    $\Rightarrow$ works fine, because transactions provide ACID properties
\end{frame}
%%%%%%%%%%%%%%%%%%%%%%%%%%%
%%%Current Implementatin%%%
%%%%%%%%%%%%%%%%%%%%%%%%%%%
  \begin{frame}
    \frametitle{Current Implementation (Control.Concurrent.STM)}
    \begin{itemize}\setlength\itemsep{1em}
      \item \textit{writeTVar}, \textit{readTVar} and \textit{newTVar} modify TVars
      \item \textit{retry} and \textit{orElse} alter the control flow
      \item \textit{atomically} executes a transaction 
      \item composition via bind operator (or do)
    \end{itemize}
  \end{frame}
  
  \begin{frame}
   \frametitle{Modify Operations}
   \begin{itemize}\setlength\itemsep{1em}
    \item \textbf{newTVar}: creates a new, initialized TVar 
    \item \textbf{writeTVar}: add the TVar and the value to the write set
    \item \textbf{readTVar}: add the TVar and its version number to the read set
    \item read set and write set serve as local buffer 
  \end{itemize}
  \end{frame}
  
%   \begin{frame}
%    \frametitle{Control Flow Operations}
%    \begin{itemize}\setlength\itemsep{1em}
%     \item \textbf{retry}: suspends until a read TVar is written.
%     \item \textbf{orElse}: takes two transactions
%       \begin{itemize}
%        \item runs the first transaction
%        \item if the first transaction retrys it runs the second transaction
%       \end{itemize}
%   \end{itemize}
%   \end{frame}
%   
  \begin{frame}
   \frametitle{\lstinline{atomically :: STM a -> IO a}}
   \begin{enumerate}\setlength\itemsep{1em}
    \item compute the read set and write set
    \item lock TVars
    \item validate the read set
    \item if valid then commit the write set
    \item else roll back
   \end{enumerate}
  \end{frame}

  \begin{frame}
   \frametitle{Validation}
    \begin{enumerate}\setlength\itemsep{1em}
%       \item lock the TVars in the read set
      \item compare version number in read set with actual version number
      \item if there is a difference then return invalid 
      \item else return valid
    \end{enumerate}
  \end{frame}

%%%%%%%%%%%%%%%%%% 
%%%%%Problems%%%%%
%%%%%%%%%%%%%%%%%%

    \begin{frame}[fragile]
    \frametitle{Problem}
    \fboxsep=0pt
    \noindent
    \begin{minipage}[t]{0.48\linewidth}
      Thread 1:
    \begin{figure}
     \begin{lstlisting}[frame=single]
a1 <- readTVar acc1
a2 <- readTVar acc2
writeTVar acc1 (a1 - 50)
writeTVar acc2 (a2 + 50)
     \end{lstlisting}
    \end{figure}
    \end{minipage}%
    \hfill%
    \begin{minipage}[t]{0.48\linewidth}
      Thread 2:
    \begin{figure}
     \begin{lstlisting}[frame=single]
a1 <- readTVar acc2
a2 <- readTVar acc1
writeTVar acc2 (a1 - 50)
writeTVar acc1 (a2 + 50)
     \end{lstlisting}
    \end{figure}
    \end{minipage}
    \vfill
    \pause
    $\Rightarrow$ either sequential or one transaction is rolled back
\end{frame}


    \begin{frame}[fragile]
    \frametitle{Idea}
    \fboxsep=0pt
    \noindent
    \begin{minipage}[t]{0.48\linewidth}
      Thread 1: 
          \begin{figure}
     \begin{lstlisting}[frame=single]
a1 <- readTVar acc1
writeTVar acc1 (a1 - 50)
a2 <- readTVar acc2
writeTVar acc2 (a2 + 50)
     \end{lstlisting}
    \end{figure}
    \end{minipage}%
    \hfill%
    \begin{minipage}[t]{0.48\linewidth}
      Thread 2:
          \begin{figure}
     \begin{lstlisting}[frame=single]
a1 <- readTVar acc2
writeTVar acc2 (a1 - 50)
a2 <- readTVar acc1
writeTVar acc1 (a2 + 50)
     \end{lstlisting}
    \end{figure}
    \end{minipage}
    \vfill
    \pause
    $\Rightarrow$ delay the evaluation of readTVar to avoid rollback, but...
\end{frame}
  
  \begin{frame}[fragile]
   \frametitle{Idea does not work}
   \begin{lstlisting}[language=Haskell]
limitedTransfer src dst am = do
   a1 <- readTVar src
   if a1 < am
     then return ()
     else do a2 <- readTVar  dst
             writeTVar src (a1 - am)
             writeTVar dst (a2 + am)
   \end{lstlisting}
   \vfill
    ... the transaction may need the value
\end{frame}

  \begin{frame}[fragile]
   \frametitle{Problem}
      \begin{itemize}\setlength\itemsep{1em}
       \item\lstinline[language=Haskell]{(>>=) :: STM a -> (a -> STM b) -> STM b}
       \item Bind extracts the value from the STM context
       \item STM no longer controls this value.
       \item Need another Typeclass than Monad
      \end{itemize}
\end{frame}


  \begin{frame}[fragile]
   \frametitle{Applicative}
      \begin{itemize}\setlength\itemsep{1em}
       \item Applicative is less powerfull than Monad
       \item\lstinline[language=Haskell]{(<*>) :: STM (a -> b) -> STM a -> STM b}
       \item The value can be modified without leaving the STM context
      \end{itemize}
\end{frame}

%%%%%%%%%%%%%%%%%%%%%%%%%%
%%%%%What I have done%%%%%
%%%%%%%%%%%%%%%%%%%%%%%%%%
  
  \begin{frame}
   \frametitle{Project}
   \begin{itemize}\setlength\itemsep{1em}
    \item Improved a pure Haskell implementation
    \item Direct notification
    \item Explicit, ordered locking
    \item Optimisations
   \end{itemize}
  \end{frame}

  
  \begin{frame}
   \frametitle{Master Thesis until now}
   \begin{itemize}\setlength\itemsep{1em}
    \item Implementation based on Applicative 
    \item Works without rollbacks...
    \item ...and without branches
    \item Constraints on TVars to gain some control
    \item Still no \textit{retry} or \textit{orElse}
   \end{itemize}
  \end{frame}


  \begin{frame}
   \frametitle{Master Thesis until now}
   \begin{itemize}\setlength\itemsep{1em}
    \item Composition by combination of Monad and Applicative
    \item\lstinline{>>=} evaluates and enables rollbacks
    \item\lstinline{<*>} enables to modify values without rollbacks
    \item \lstinline[language=Haskell]{writeTVar :: TVar a -> STM a -> STM ()} 
   \end{itemize}
  \end{frame}

%   \begin{frame}[fragile]
%    \frametitle{\lstinline[language=Haskell]{writeTVar :: TVar a -> STM a -> STM ()}}
%    \begin{lstlisting}[language=Haskell]
% example = do
%    writeTVar t2 (readTVar t1)
%    a <- readTVar t2
%    if a ...
%    \end{lstlisting}
% \end{frame}

  
  \begin{frame}
   \frametitle{New Combinators}
   \begin{itemize}\setlength\itemsep{1em}
    \item \lstinline[language=Haskell]{(<*>) :: STM (a -> b) -> STM a -> STM b}
    \item \lstinline[language=Haskell]{(<**>) :: STM a -> STM (a -> b) -> STM b}
%     \item \lstinline[language=Haskell]{(*>) :: STM a -> STM b -> STM b}
    \item \lstinline[language=Haskell]{(**>) :: STM a -> (STM a  -> STM b) -> STM b}
    \item \lstinline[language=Haskell]{(>>=) :: STM a -> (a -> STM b) -> STM b}
    \item \lstinline[language=Haskell]{(>>) :: STM a -> STM b -> STM b}
   \end{itemize}
  \end{frame}

  
  \begin{frame}[fragile]
    \frametitle{New Transfer}
    \lstinputlisting{ressources/accountTVar2.hs}
\end{frame}
  
 
  \begin{frame}[fragile]
    \frametitle{Problem solved}
    \fboxsep=0pt
    \noindent
    \begin{minipage}[t]{0.48\linewidth}
      Thread 1:
      \begin{figure}
       \begin{lstlisting}[frame=single]
atomically $ 
  transfer acc1 acc2 50
       \end{lstlisting}
      \end{figure}
    \end{minipage}%
    \hfill%
    \begin{minipage}[t]{0.48\linewidth}
      Thread 2:
      \begin{figure}
       \begin{lstlisting}[frame=single]
atomically $ 
  transfer acc2 acc1 50
       \end{lstlisting}
      \end{figure}
    \end{minipage}
    \vfill
    $\Rightarrow$ no more rollback
\end{frame}
  
  
%%%%%%%%%%%%%%%%%%%%%%
%%%%%Future Work%%%%%%
%%%%%%%%%%%%%%%%%%%%%%
  \begin{frame}
   \frametitle{Todo}
   \begin{itemize}\setlength\itemsep{1em}
    \item Reduce the number of combinators
    \item Change writeTVars type and use ApplicativeDo
    \item \textbf{But} it extracts the Value from STM context
    \item investigate other problems:
      \begin{itemize}
        \item Branch condition is not changed by TVar modification
        \item Recomputation of values which did not change
      \end{itemize}
   \end{itemize}
  \end{frame}

  \begin{frame}[fragile]
    \frametitle{Unnecessary Recomputation}
    \begin{lstlisting}[language=Haskell]
     transaction = do
       lmitedTransfer acc1 acc2 50
       limtedTransfer acc3 acc4 100
    \end{lstlisting}
    \vfill
    If one transfer is invalidated, both are recomputed
\end{frame}


\begin{frame}
 \frametitle{Questions about...}
 \begin{itemize}\setlength\itemsep{1em}
  \item ...Control.Concurrent.STM?
  \item ...rollback avoidance?
  \item ...unnecessary recomputation?
  \item ...something else?
 \end{itemize}

\end{frame}


  
\end{document}



