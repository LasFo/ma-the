\chapter{Conclusion}

\label{Chapter5}

This chapter contains related work, future work and a summary of the thesis.

\section{Related Work}
The most important work for Haskell regarding STM is the paper of \parencite{STMBase}. This work forms the foundation 
of the current implementation of STM in GHC. Even if the original implementation was reworked over the 
last years, its core is still present. This core is described in Chapter \ref{Chapter1}. The modifications
of the original library were in most cases bug fixes. The only feature that was added to initial
implementation are \keyword{invariants} \parencite{invariants}. Besides an implementation sketch
and a description of the interface, \parencite{STMBase} also offer a formal semantics for STM in Haskell.
The alternative implementation implements the same interface and aims to fulfill these
semantics. There is no formal prove that the alternative implementation (or the GHC implementation) 
suffice these semantics. 

Another approach to optimize STM is presented by \parencite{pessimisticSTM}. They propose to 
use a pessimistic approach comparable to data base transactions. This means each time a shared data structure 
is accessed it is locked. If the transaction tries to acquire a locked data structure it rolls back. 
This design avoids \textit{shadow copies} of values such as the logs in the Haskell implementations. This avoids 
a memory overhead and more important it avoids the necessity to look up values in the log. They made 
assumptions for the ussage of transactions: Most transactions commit successfully and they use far more reads
than writes. Under these assumptions their implementation performs considerable better than optimistic 
implementations. In the course of this thesis, we did not investigate how a pessimistic implementation 
performs. Neither made we any assumptions on the usage of STM. The aim was to create a implementation
that performs better in general. The timing tests in Chapter \ref{Chapter4} show that this was not achieved. 
There are cases where the alternative implementation performs better than the original implementation
and vice versa. In this regard the thesis results are comparable to the results of \parencite{pessimisticSTM}. 
Under the assumption that the transactions are sufficient expensive and the level of concurrency is 
suffiecent high the new implementation is faster. 

In Chapter \ref{Chapter1} we defined the current problems of the STM implementation in the GHC. One problem
is \textit{when} transactions are rolled back. This problem was solved with this thesis. The other problem is
\textit{how} transactions are rolled back. This thesis contains no solution for this problem, but \parencite{checkpoint}
present a draft that engages this problem. They present a concept that automatically creates 
\keyword{checkpoints}. If a transaction is rolled back, it is not restarted from the very beginning. The 
systems validates the checkpoints and restarts the execution from the latest valid checkpoint.
The checkpoints basically consist of a log and a list of remaining operations. In theory, it is useful
to create a checkpoint for every read operations on shared data structures. This is not recommended in practice since
every creation of a checkpoint consumes time and memory. Thus if the systems creates the maximum amount 
of checkpoints it never executes operations needlessly multiple times, but its overhead will
most likely revokes the performance benefits. The solution is to group action and create checkpoint 
for these grouped actions. To find an optimal group size remains an open problem.

\section{Future Work}
Most of the functions provided by GHC STM are also provided by the alternative implementation presented in
this thesis. Nevertheless, there are parts of the external interface and internal semantics that are not 
covered by the new implementation. This lack of functionality offers the opportunity for further research.

In Chapter \ref{Chapter3} the \code{globalCount} were introduced which is used to create globally unique 
\code{ID}s. \code{ID} is a type synonym for \code{Int}. In Haskell \code{Int} is a 32 or 64 bit integer. Thus 
the number of \code{ID}s is limited. The only usage of \code{ID} is to create new TVars. For terminating 
applications this restriction can be neglected. If the user does not create and delete TVars all the time,
he will run out of memory before \code{globalCount} overflows. If this STM implementation is used within an
infinite running systems such as web servers, it may be an issue (assuming that the systems creates and
deletes TVars while running). This problem can be avoided by either using an \code{Integer} which presents
an integer of unlimmited size in Haskell or by using a kind of conflict detection to be able to handle the 
overflow. The first solution adds additional overhead to the implementation, since \code{Int} is slightly 
faster than \code{Integer}. Furthermore, we need to alter the data types used within the implementation. 
We used \code{IntMap} for the internal book keeping of the alternative implementation. An \code{IntMap} cannot 
handle \code{Integers}. The overflow detection can be used in two ways. If the implementation is able 
to reclaim unused \code{ID}s, it can reuse these. The conflict detection can also terminate the execution to 
avoid a miss behaviour of the implementation. Currently the implementation does not detect the overflow 
and thus may lead to problems in long running systems that use STM.

In Chapter \ref{Chapter4} we discovered that STMWSL perfoms poorly in the performance tests. The reasons for this are 
that the implementation can roll back even if the transaction does not branch. Additionally does the 
transaction perform two validations to avoid unnecessary locking. The locking is performed by taking
the MVars that hold the value for a TVar. By taking this MVar no one is able to read these TVars which 
is important for the protocol. Unfortunately, the transaction that took these MVars is also not able to read 
the associated TVars. Thus the transaction must evaluate the delayed reads before it acquires the locks of these 
TVars. This contains the danger that the TVars are modified after the transaction read them and before
it acquires the locks for them. This results in a rollback. If the lock holding transaction were able 
to read the TVars, it could evaluate the delayed reads after it acquired the locks and validated. On the
one hand, this makes the validation less expensive since only the reads that were evaluated in the 
computation phase are validated. Furthermore, this avoids rollbacks because the values cannot 
be modified in a manner that it evokes a rollback. After the transaction t1 reads the TVars, it is 
still possible that another transaction modifies the values that are only read by t1, but not modified.
This is not a violation of the ACI properties and thus would not cause a rollback (see \parencite{lockfreedom}).
In conclusion, another locking mechanism that allows the lock holder to read the locked structure
increases the performance of STMWSL significantly. This also adds more complexity to the implementation
because reading in a locked context contains the risk of deadlocks. Avoiding these deadlocks adds 
additional complexity to the implementation.

The alternative implementation lacks a sufficient support for exception handling. If an exception is
raised, this exception is thrown to the caller. This behaviour is not incorrect per se. If the 
exception is raised because the transaction has seen an inconsistent view of the memory, it is 
incorrect. When the transaction raises an exception during the execution, the implementation must 
check whether the transaction is valid or invalid. If it is invalid the transaction have to be rolled 
back, otherwise it is a violation of the consistency property. If the transaction is valid, the 
behaviour is a design decision. Either the transaction is raised to the caller or the transaction 
handles the exception like a retry, meaning it waits for a change of the read TVars and restarts.
Both execution paths preserve the ACI properties and thus are legit. There are additional 
technical problems with exceptions in the implementation. If an exception is raised during the 
commit phase, no measures are performed to make sure the system 
is still usable. If the transaction already locked several TVars and then raises an exception, this 
exception is not catched and the locks for the TVars are not released. Thus it is possible that 
the TVars remain locked. I am not sure if it is possible that an exception is raised at this point
because only read and write operations on MVars and safe operations on the \code{IntMaps} are performed.
Nevertheless, it remains an unsolved problem.

\parencite{invariants} introduce a new feature to STM called \keyword{invariants}.
This allows the user to define invariants on the TVars that must be fulfilled before a transaction
can commit. If a transaction tries to commit a state that violates at least one invariant the 
transaction is rolled back (or suspended in case none of the read TVars has changed). This allows
the user to compose STM functions even better for the costs of performance. An integration of 
invariants in the alternative implementation inceases the usability of the alternative implementation
to a level that is similar to GHC STM.

STMLA has proven to be competitor to the GHC implementation of STM with regards to performance, even 
though the implementation makes no use of low level C primitives to boost its performance. There are 
cases where the GHC implementation performs better. To implement STMLA 
with C primitives and integrating it in the compiler provides multiple benefits. The performance 
increases could be enough to substitue the current implementation in GHC. This need to be investigated
with additional performance test. Like stated in Chapter \ref{Chapter4}, this is a difficult topic,
since STM is a universial tool and cannot be tested easily. Even if the implementation is not fast
enough to replace the current implementation, it can provide an alternative to the current implementation
which is used for specific cases. Then the user (or compiler) can decide which implementation he wants to use. 
More importantly the integration
in the compiler allows the implementation to detect conflict before the commit phase. This is 
important if the transaction executes an infinite loop because it saw an inconsistent view of the 
memory. Without an early conflict detection the transaction does not break the loop because it never
validates before the commit phase. The GHC implementation uses the runtime system to evoke validations
in the computation phase to avoid these kinds of loops. A compiler integration of the alternative 
implementation allows a similar mechanism.

We introduced in Section \ref{Prob:UnRe} another problem of the GHC implementation \footnote{This 
Problem is also present in the Project implementation and the alternative implementations.}: the way 
transactions are rolled back. Regardless the reason for a rollback, the transaction starts all
over. This includes that the transaction executes actions on TVars that have not changed. Unfortunately,
I was not able to implement a suitable solution for this problem in the time of this thesis. The previously
presented paper \parencite{checkpoint} had some success engaging this problem. To provide a similar 
solution for Haskell may be possible by redefining the monadic operations \code{>>} and \code{>>=}. 
The first step is to divide the transaction in the parts that are seperated by \code{>>} or \code{>>=} 
(the single \textit{statements} of the do-notation). We call these parts chunks for now. The second step 
is to deteremine which TVars are used in which chunks. \code{unsafePerformIO} can be used to do this. 
\code{>>} and \code{>>=} both execute the first action. During this execution \code{unsafePerformIO} can
be used to detect the values (and the associated TVars) that are needed to evaluate this chunk. In the
end the transaction contains an ordered collection of pairs. Each pair consist of a chunk and a set of
TVars that this chunk depends on. In the case the transaction is invalid it can be rolled back to the 
newest chunk that is still valid. This includes that every time the transaction logs a chunk and its 
dependencies, the transaction also needs to backup the read/writeLog. Thus this implementation contains
a lot of book keeping. It needs further research to decide whether this idea works correctly and if so
whether it increases the performance despite the book keeping overhead.

\section{Summary}
In Chapter \ref{Chapter0} we gave a motivation for synchronization in general. To utilize multi-core
processors and guarantee reactivity is mandatory for software nowadays. Multithreading is essential 
to achieve these objectives. Multithreading also includes dangers such as lost updates or race conditions. 
These problems are engaged with synchronization tools, starting from semaphores to abstract concepts
such as STM.

Chapter \ref{Chapter1} first motivated STM in Haskell by comparing it to MVars. The usage of TVars 
and MVar is similar, but TVars or more specifically STM guarantee the ACI properties which makes
deadlock avoidance much easier to handle. Furthermore, we explored the current implementation of STM
in GHC. This library makes use of low level C primitives and the runtime systems to ensure its 
correct behaviour and performance. At last, we examined the performance problems in this implementation
and introduced the idea of delaying IO-reads to avoid one problem. The problems are \textit{how} and 
\textit{when} a transaction is rolled back. This also definied the aims of the thesis, i.e. 
to provide an implementation that avoids these problems. 

The concept to provide an implementation that avoids one of these problems is presented Chapter \ref{Chapter2}.
First we identified the technical reason for unnecessary rollbacks. We introduced the term
\keyword{critical TVars} to denote TVars whos modification results in a rollback. Furthermore, we 
specified when it is mandatory to evaluate an IO-read; if the value is used in a branch condition.
The key idea is to minimize the time the TVars are critical. This is achieved by delaying the 
evaluation of IO-reads as far as possible. The result is that IO-reads, that are not needed for branch 
conditions, are delayed to the commit phase and others are evaluated just before they are used.

The main part of this thesis is presented in Chapter \ref{Chapter3}. The implementation of the previously 
presented idea is described in detail in this Chapter. The STM monad is a state monad and processing a
transaction is devided in two phases, the computation phase and the commit phase. In the computation
phase the state is enriched by the \code{writeTVar} and \code{readTVar} operations. In the commit phase 
this state is used to validate the transaction an possibly publish its results. To delay the execution 
of the IO-read and to find out when a value is needed for a branch condition, the values are wrapped 
by an \code{unsafePerformIO} action. If this action is not processed in the computation phase, its 
execution is forced in the commit phase to ensure the reads are evaluated before the transaction 
finishes. 

The new implementation is compared to the original implementation and a reference implementation in
Chapter \ref{Chapter4}. The results of these performance tests look promising. The new implementation
is in all tests faster than the reference implementation and in some test even faster than the GHC implementation.
Nevertheless, more tests are needed to verify if the implementation is faster in general. The expectations 
regarding the execution time, when increasing the level of concurrency, are also met. If we increase the 
level of concurrency and no TVars are critical, the alternative implementation does not lose performance.

\subsubsection{Conclusion}
The aim of this thesis was to provide an alternative implementation of STM in Haskell to avoid the 
problems of unnecessary rollbacks and unnecessary recomputations. The presented implementation 
is a pure Haskell implementation in contrast to the original implementation that uses C primitives.
The alternative implementation uses \code{unsafePerformIO} to avoids the unnecessary rollbacks. 
The previous section discussed an idea to solve the problem of unnecessary recomputations. However, 
this idea is not implemented and tested in the alternative implementation. The test results have 
shown that the new implementation performs better than a reference implementation and equally to 
the original implementation. Besides the missing support for exception handling and invariants, 
the alternaive implementation provides the same functions as the original implementation. 