% Appendix A

\chapter{Applicative} % Main appendix title

\label{AppendixA} % For referencing this appendix elsewhere, use \ref{AppendixA}
In this Appendix we will explore another idea to solve the Problem of the unnecessary rollbacks.

The thesis started with the observation that the rollback explained in section \ref{Prob:UnRo} is 
unnecessary. It is a modify operation that does not depend on the actual value of the TVar. Thus 
read the value, modify it and write it back is a imperative way of solving this task. In other 
words, it is unnecessary to read the TVar during the computation phase if the transaction does
not branch on that value. At this point we did not see that it is not needed when its evaluation is
not demanded. In order to avoid the rollback, we wanted a mechanism that prohibits the user from 
branching depending on the value. The user should still be able to use that value to do pure computations and 
write it to TVars. This allows the library to decide when the IO-read is evaluated. We decided to 
use Applicative\footnote{https://hackage.haskell.org/package/base-4.9.0.0/docs/Control-Applicative.html}.
Applicative is comparable to Monad, but less powerfull since no operation like \code{>>=} is usable.
Applicative prevents us from extracting the value from its context.
Additionally did we change the type of \code{writeTVar :: TVar a $\rightarrow$ STM a $\rightarrow$ STM ()}. 
This allows the following
example: 
\begin{lstlisting}
inc tv = writeTVar tv $ 
         pure (+1) <*> 
         readTVar tv
\end{lstlisting}
We are able to modify the value of a TVar without \code{>>=}. Furthermore did we changed the interal
semantics, that an IO-read is performed either by \code{>>=} or in the commit phase. It is not performed
in \code{readTVar}. Thus if we read a TVar and use applicative operations instead of monadic operations,
we do not risk a rollback. The value is evaluated in the commit phase, hence it is not critical at all.
This approach avoids the intially mentioned rollback, but has its drawbacks. 

The use of applicative operations is not as comfortable as monadic operations. One problem is the use of 
multiple operators instead of just one. There is no suitable language feature such as the do-notation like 
in the monadic version. We cannot use \code{ApplicativeDo} because the target
of \code{ApplicativeDo} is something else\footnote{\code{ApplicativeDo}s aim is it to use applicative 
operations to calculate an result and not to process a chain of operations. To understand the problem,
we would need to understand the translation scheme of \code{ApplicativeDo} which goes beyond the scope of
this thesis.}. Consequently, we need to use brackets or \code{\$} to group the actions. Another problem is  
the reversed order of operations. In the monadic version, the operations are ordered like in an imperative
language, for example: read $\rightarrow$ (+ 1) $\rightarrow$ write. This is easily comprehensible. The 
applicative version uses a functional style, for example: write (1 + (read)). For this example it is 
comprehensible, but you can imagine what happens if the actions increase in size. For every operation
we introduce in the applicative version another nesting. This leads very fast to incomprehensible code,
especially when using functions with more than one argument. The monadic version is just a chain of operation, 
which can easily be extended. 

In conclusion we gain performance for the costs of usability of the library; one of the biggest strength of
STM. We solved one problem by introducing another problem. Nevertheless, I implemented a version of STM
that uses this scheme and avoided the targeted rollbacks. \code{IO ()} actions were used
to delay the evaluation\footnote{This implementation did not use \code{unsafePerformIO}. It used monadic \code{IO} actions.}. 
When \code{>>=} is executed these \code{IO} actions are evaluated. The \code{IO} actions are created by \code{readTVar}.
Remember that \code{readTVar} returns an STM action. This action can either be used to bind its result with \code{>>=} or it 
can be used to write another TVar (the value may be modified after read and before written). If it is written to a 
TVar the \code{IO} action was logged and executed in the commit phase. This solution contained some problems.

The question arised how we are able to execute an action and use it at the same time. We need to execute it to make sure 
the correct value is the result of the action. This can either be the actual value of the TVar in a form of an \code{IO} 
action or it is an \code{IO} action that is written in the log. The following example would not work:
\begin{lstlisting}
swap t1 t2 = do
  let a = readTVar t1
  writeTVar t1 <*> readTVar t2
  writeTVar t2 a
\end{lstlisting}
This swap function does not work because the \code{readTVar} operation in the let expression is evaluated when 
\code{writeTVar t2 a} is evaluated. At this point the value of \code{t1} is no longer its initial value, but the 
value of \code{t2}. Hence, in the end of the transaction both TVars contain the same value. To solve this problem
we introduced the function \code{eval :: STM a -> STM (STM a)}. This funcion allows us to evaluate the 
\code{STM} action and use it at the same time. It first evaluates the \code{STM} action and then returns a 
dummy action that has no side effects\footnote{Side effects in this context are modifications of the \code{StmState}}. 
The return value of this is that of the passed action at the point \code{eval}
was called. For example:
\begin{lstlisting}
swap t1 t2 = do
  act <- eval $ readTVar t1
  writeTVar t1 <*> readTVar t2
  writeTVar t2 act
\end{lstlisting}
This swap function is correct. The first action evaluates \code{readTVar t1} and creates an action that return 
the same as \code{readTVar t1} at this point. This works for every \code{STM} action. Side effects evoked by the 
action on the other hand are not evoked by the returned action. However, the \code{eval} function also solved another problem. The following
example is not possible without eval:
\begin{lstlisting}
double t1 t2 t3 = do 
  act <- eval $ readTVar t1
  writeTVar t2 act
  writeTVar t3 act
\end{lstlisting}
It is not possible to write this example without evaluating the \code{readTVar t1} twice or using \code{eval}. 
\code{eval} can be compared to \code{share}\footnote{https://hackage.haskell.org/package/explicit-sharing-0.9/docs/Control-Monad-Sharing.html}
whichs allows explicit sharing for monadic action results.

Even though \code{eval} allows us to use a readTVar action twice, there are still problems. The previous example
creates log entries for \code{t2} and \code{t3} that both yield the same \code{IO} action. The \code{IO} 
action that IO-reads \code{t1}. In the commit phase these action are evaluated. This includes that the actual 
TVar is read twice because the action is present in two log entries. At the first glance it does not seem to
be an issue, but also pure computation are duplicated as the following example emphasises:
\begin{lstlisting}
shareless t1 t2 t3 = do 
  act <- eval $ pure sqrt <*> readTVar t1
  writeTVar t2 act
  writeTVar t3 act
\end{lstlisting}
In this example the pure function \code{sqrt} is also evaluated twice. This is necessary in general because the 
IO-read (created by readTVar) could return different values depending on the time it is executed. For our case
we know that this cannot happen. Both IO-reads are performed in the commit phase. At this time the TVar \code{t1} is 
locked. Thus the actual value of \code{t1} cannot change (as long as we do not change it ourself). 

The core problem is that Haskell does not allow sharing on \code{IO} actions drirectly. There is a way to enable 
sharing on \code{IO} actions: \code{unsafePerformIO}. These actions are considered as pure values, 
hence their results are shared as any other pure value in Haskell. While using using \code{unsafePerformIO}
to implement the sharing in the applicative solution, we discovered that it can replace the whole need for 
Applicative. Since the usability of the applicative solution was still worse than the usability of the 
original implementation, we discarded the applicative implementation and created the implementation that
is presented in the main part of this thesis.


